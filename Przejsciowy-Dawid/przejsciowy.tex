%% Libs
\documentclass[polish,11pt,a4paper]{article}
\usepackage[a4paper, margin=2cm]{geometry}
\usepackage[T1]{fontenc}
\usepackage{babel}
\usepackage{graphicx}
\usepackage{ragged2e}
\usepackage{caption}
\usepackage{amsmath} 
\usepackage{amssymb} 
\usepackage{setspace}
\usepackage[utf8]{inputenc}
\usepackage{subfig}
\usepackage{cancel}
\usepackage{import}
\usepackage{svg}

%% start document
\begin{document}
	
	%% Strona tytułowa
	\setstretch{1.5}
	\centering
	\section*{PRACA PRZEJŚCIOWA}
	\section*{Projekt oprogramowania do matematycznego modelowania układów niecałkowitego rzędu}
	
	\large
	KOD PRZEDMIOTU: MYAR2S22001M
	\break
	\large
	\break
	\break
	\break
	
	\raggedright
	Autor: Ostaszewicz Dawid
	\break
	
	Kierunek: Automatyka i Robotyka, II stopień

	Prowadzący: dr inż. Tomasz Huścio
	\clearpage
\justifying
\section*{Cel projektu}
Celem projektu jest opracowanie oprogramowania do matematycznego modelowania układów niecałkowitego rzędu. Badania nad układami tego rodzaju z roku na rok przyciągają coraz szersze grono ekspertów z dziedziny automatyki i teorii sterowania. Taka sytuacja ma miejsce ze względu na zdecydowanie poprawione warunki dynamiczne rachunku całkowo-różniczkowego niecałkowitego rzędu, względem klasycznych układów całkowitych.
\section*{Analiza literatury}
Rachunek ułamkowego rzędu ma swoje początki już w 1832 r. Badania wybitnego matematyka Josepha Riemanna doprowadziły do powstania generalizacji silni z argumentów będących liczbami naturalnymi do liczb rzeczywistych. Uogólniona funkcja nazywa się funkcją gamma i jest aproksymacją, która zwiększa dziedzinę wartości funkcji silni. Na przestrzeni ostatnich stu lat powstało wiele definicji rachunku ułamkowego rzędu. Główny trzon stanowią definicje Riemann'a-Liouvill'a, która jest stosunkowo prosta od obliczenia i przekształcania, jednak nie można w niej założyć warunków początkowych. Następną popularną definicją jest różniczka Caputo, jest odrobinę trudniejsza w przekształceniach, jednak pozwala na badanie rachunku pod względem różnych warunków początkowych. Trzecią najczęściej stosowaną definicją jest definicja dyskretna Grünwald'a-Letnikov'a. Znaczącą cechą rachunku jest posiadanie pamięci, która w różnych definicjach różnie współdziała wraz z postępem czasu i w inny sposób wpływa na przyszłe działanie układu. Podstawowe definicje i matematyczną stronę rachunku omawia prof. Podlubny [37]. Prof. Kaczorek wraz z dr. Rogowskim opisują zastosowanie rachunku w układach odnoszących się do teorii sterowania [21],[22]. Główny trzon badań nad matematyką i implementacją mikroprocesorową wymienianego rachunku prowadzi również prof. Ostalczyk [33]. Obecnie istnieje kilka bibliotek Matlab'a do wykonywania symulacji układów niecałkowitego rzędu. W tej dziedzinie wiodą FOMCON, FOTF [6], [42], [43]. Definicję dyskretną w oprogramowaniu stosował prof. Ostalczyk [11]. Rozwój rachunku i metod numerycznych prowadzi do rozwoju bardziej zaawansowanych technik sterowania i regulacji obiektów dynamicznych, jak również pozwala na lepszy opis dynamiki wybranych obiektów [3], [7], [10], [14], [26], [27], [44]. Nie ma ogólnodostępnej biblioteki, która pozwala na symulację układów z definicji  Grünwald'a-Letnikov'a, z tego względu takie oprogramowanie powstanie w tym projekcie.
\nocite{*}  % This command includes all entries from the .bib file

\bibliographystyle{plain}
\bibliography{biblioteka}
	
\end{document}
